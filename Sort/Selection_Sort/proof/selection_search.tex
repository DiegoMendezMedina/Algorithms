\documentclass[12pt, twoside, a4paper]{article}
\usepackage{hyperref}
\usepackage[dvipsnames]{xcolor}
\definecolor{rosamexicano}{rgb}{1.0, 0.01, 0.24}
\hypersetup{
  colorlinks = true,
  filecolor = red,      
  urlcolor = blue,
}
\begin{document}
\begin{center}
  {\large \bfseries Selection Sort Algorithm \par}
\end{center}
Consider sorting $n$ numbers stored in array A by first finding the smallest element of $A$ and exchanging it with element in $A[1]$. Then find the second
smallest element of $A$, and exchange it with $A[2]$. Continue this manner
 for the first $n-1$ elements of $A$.\\
 You can find the implementation \href{https://github.com/DiegoMendezMedina/Algorithms/tree/master/Sort/Selection_Sort/implementations}{\textbf{here}}
 or go to the next url: \url{https://github.com/DiegoMendezMedina/Algorithms/tree/master/Sort/Selection_Sort/implementations}. \\ \\
  \begin{center}
    {\large \textbf{Pseudocode} \par}
  \end{center}
  SELECTION-SORT(A)
  \begin{enumerate}
  \item \textbf{for} i = 1 \textbf{to} n-1
  \item \hspace{.5cm} smallest = A[i]
  \item \hspace{.5cm} k = i
  \item \hspace{.5cm} \textbf{for} j = i \textbf{to} n
  \item \hspace{1cm} \textbf{if} A[j] $<$ smallest
  \item \hspace{1.5cm} smallest = A[j]
  \item \hspace{1.5cm} k = j
  \item \hspace{.5cm} \textbf{if} i $\neq$  k
  \item \hspace{1cm} $change(a, i, k)$
  \end{enumerate}
  CHANGE(A, b, c)
  \begin{enumerate}
  \item aux = A[b]
  \item A[b] = A[c]
  \item A[C] = aux
  \end{enumerate}
  \newpage
  \begin{center}
        {\large \textbf{Proof} \par}
  \end{center}
  \textbf{Loop invariant:}\\ At the start of each iteration of the \textbf{for} loop (lines 1-9), $A[1..i]$ is sorted. i.e the i-th element of the array is
  greater for every previous numbers on the array.\\
  On the second \textbf{for} loop (lines 4-7), the smallest number in A[i..n] is found \textbf{if} it's different from the one in the $i-$th position there's
  a $change$. So now A[1...i+1] is also sorted.\\ \\
  \textbf{Initialization:} \\ 
  When $i = 1$ there are no previous \textbf{i} values.\\
  The smallest number in A[1..n] is found and is changed with the one in
   A[1]. Now A[1,2] is sorted. \\ \\
  \textbf{Maintenance:} \\
  There's another iteration which means that $A[1..i]$ is sorted. \\
  The smallest number in A[i..n] is found and is changed with the one in
  A[i]. Now A[1...i+1] is sorted. \\ \\
  \textbf{Termination:}\\
  $A[i...n-1]$ is sorted and $A[n]$ is greater than $A[n-1]$,
  hence $A[1...n]$ is sorted.\\
\end{document}
